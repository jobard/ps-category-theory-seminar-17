\documentclass{beamer}

\usepackage[utf8]{inputenc}

\beamertemplatenavigationsymbolsempty
\setbeamertemplate{footline}{
	\raisebox{0.5em}{\makebox[\paperwidth-0.5em][r]{
		\color{gray}\scriptsize\insertframenumber}}}

\usepackage{mathtools}
\usepackage{tikz}
\usetikzlibrary{cd, shapes, tikzmark}
\def\pathToRoot{../../}\usepackage{nag}
\makeatletter
\@ifclassloaded{beamer}{}
{\usepackage[small,compact]{titlesec}}
\makeatother
\usepackage[utf8]{inputenc}
\usepackage[T1]{fontenc}
\usepackage{lmodern}
\usepackage{color}
\usepackage{parskip}
\usepackage{needspace}
\usepackage{microtype}
\usepackage{mathtools}
\usepackage{xifthen}
\usepackage{xpatch}
\usepackage{enumitem}
\usepackage{mdwlist}
\usepackage{bussproofs}
\EnableBpAbbreviations
\usepackage{tabu}
\usepackage{amssymb}
\usepackage{amsmath}
\usepackage{amsthm}
%grober hack, der den groben hack von parskip bei den amsthm sachen korrigiert
\begingroup
    \makeatletter
       \@for\theoremstyle:=definition,remark,plain\do{%
            \expandafter\g@addto@macro\csname th@\theoremstyle\endcsname{%
                        \addtolength\thm@preskip\parskip
             }%
        }
\endgroup
\usepackage[UKenglish]{babel}
\usepackage{xparse}
\usepackage{adjustbox}
\usepackage{geometry}
\usepackage{booktabs}
\usepackage{multicol}
\usepackage{soul}
\usepackage{calc}
\usepackage{textcase}
\usepackage{stmaryrd}
\usepackage{marvosym}
\usepackage{wasysym}
\usepackage{pifont}
\newcommand{\cmark}{\ding{51}}
\newcommand{\xmark}{\ding{55}}
\usepackage{tikz}
\usetikzlibrary{trees, backgrounds, shapes, chains, decorations.text, decorations.pathreplacing, circuits.logic.IEC, patterns, matrix}
\usepackage{tikz-qtree}
\usepackage{tikzsymbols}
\usepackage{fancyvrb}
\usepackage{fancyhdr}
\usepackage{verbatim}
\usepackage[framemethod=tikz]{mdframed}
\usepackage{lastpage}
\usepackage{pgfpages}
\usepackage{csquotes}
\usepackage{longtable}
\usepackage{ragged2e}
%\usepackage{stackengine}
\usepackage{censor}
\usepackage{expl3}
\usepackage{multirow}
\usepackage{hyperref}
\usepackage{environ}


% Package for Cateogry diagrams:

\usepackage{tikz-cd}



\ifcsdef{labelenumi}{
\renewcommand{\labelenumi}{(\alph{enumi})}
\renewcommand{\labelenumii}{(\roman{enumii})}
}{}

\input{\pathToRoot headers/definitions}



\tikzset{
    normal/.style={draw, semithick},
    n/.style={style=normal, circle, inner sep=1mm, minimum size=8mm},
    l/.style={style=normal, rounded corners=1mm, inner sep=1mm, minimum size=6mm},
    e/.style={style=normal, shorten >=1mm, shorten <=1mm, ->, >=stealth},
    syntax/.style={style=normal, ellipse, minimum height=6mm, minimum width=8mm}, % nodes in syntax trees
    inner/.style={style=normal, minimum size=4mm}, % inner leaves or root in normal trees
    leaf/.style={style=normal, circle, minimum size=4mm}, % leaves in normal trees
    te/.style={style=normal}, % edges in a tree
    be/.style={style=e, dashed} % binding edge
}

\newcommand{\syntaxtree}[1]{ % DEPRECATED - use tikzsyntaxtree
    \begin{tikzpicture}[baseline=(current bounding box.north)]
        \tikzset{grow=down}
        \tikzset{every node/.style={syntax}}
        \tikzset{edge from parent/.style=
            {te,
                edge from parent path={(\tikzparentnode) -- (\tikzchildnode)}}}
        \Tree #1
    \end{tikzpicture}
}

\newenvironment{tikzsyntaxtree}[1][]{
    \begin{tikzpicture}[baseline=(current bounding box.north), #1]
    \tikzset{grow=down}
    \tikzset{every tree node/.style={syntax}}
    \tikzset{edge from parent/.style={te, edge from parent path={(\tikzparentnode) -- (\tikzchildnode)}}}
}{
    \end{tikzpicture}
}


\newcommand{\DisplayScaledProof}{\maxsizebox{\linewidth}{!}{\DisplayProof}}
\newcommand{\DisplayTopProof}{\adjustbox{valign=t}{\DisplayProof}}
\newcommand{\DisplayScaledTopProof}{\adjustbox{valign=t}{\maxsizebox{\linewidth}{!}{\DisplayProof}}}


\newcolumntype{P}[1]{>{\RaggedRight\hspace{0pt}}p{#1}}

\newenvironment{prooftable}
{
    \begin{longtable}{>{\footnotesize}p{0.33\textwidth}>{\footnotesize}p{0.33\textwidth}|>{\footnotesize}P{0.15\textwidth}}
    \normalsize Textbeweis & \normalsize Erklärungen & \normalsize Schlussregel\\\hline
    \endhead
}
{
    \end{longtable}
}


\theoremstyle{definition}
\newtheorem*{definition*}{Definition} % Definition ohne Nummer
\newtheorem*{inferenceRule*}{Schlussregel}

\usepackage{semantic}

\title{Presheaf model for dependent types}

\author{Nikita Ziuzin}
\date{July 12, 2017}

\begin{document}

\frame{\titlepage}

% * Motivating example
% * Syntax and semantics from Hofmann (Slides capture all the rules from text)
% - Syntax, no identities or naturals
%   - Context semantics
%   - dependent product
%   - dependent sum, projections
%   * Context morphisms, mention that we will use generalised subst
%   * Categories with Families (Huber)
%   - Definition
%   - dependent product with CwF
%   - dependent sum with CwF
%   * Presheaf model for CwF
%   - dependent product as a presheaf
%   - dependent sum an exercise (?)

\begin{frame}
  \frametitle{Context}

  \begin{align*}
    &\vdash \Gamma \text{ ctxt} &&\text{$\Gamma$ is a valid context}\\
    &\Gamma \vdash A \text{ type} &&\text{$A$ is a type in $\Gamma$}\\
    &\Gamma \vdash a : A &&\text{$a$ is a term of type $A$ in $\Gamma$}\\
    &\vdash \Gamma = \Delta \text{ ctxt}  &&\text{$\Gamma$ and $\Delta$ are definitionally equal contexts}\\
    &\Gamma \vdash A = B \text{ type}  &&\text{$A$ and $B$ are definitionally equal types in $\Gamma$}\\
    &\Gamma \vdash a = b : A  &&\text{$a$ and $b$ are definitionally equal terms of type $A$ in $\Gamma$}\\
  \end{align*}

  \[
    \inference[C-Emp]{}{\vdash \diamond \text{ context}}
    \qquad
    \inference[C-Ext]{\Gamma \vdash A \text{ type}}{\vdash \Gamma, x:A \text{ context}}
  \]

  \[
    \inference[Var]{\vdash \Gamma, x:A, \Delta \text{ context}}{\Gamma, x:A, \Delta \vdash x:A}
  \]
\end{frame}

\begin{frame}
  \frametitle{Dependent product}

  \[
    \inference[$\Pi$-F]{\Gamma \vdash A \text{ type} \qquad \Gamma, x:A \vdash B \text{ type}}{\Gamma \vdash \Pi x:A. B \text{ type}}
  \]

  \[
    \inference[$\Pi$-I]{\Gamma, x:A \vdash b : B}{\Gamma \vdash \lambda x : A. b : \Pi x:A. B}
  \]

  \[
    \inference[$\Pi$-E]{\Gamma \vdash u : \Pi x:A. B \qquad \Gamma \vdash a : A}
      {\Gamma \vdash App_{[x:A]B}(u, a): B[a / x]}
  \]

  \[
    \inference[$\Pi$-C]{\Gamma \vdash \lambda x : A. b : \Pi x:A. B \qquad \Gamma \vdash a : A}
      {\Gamma \vdash App_{[x:A]B}(\lambda x : A. b, a) = b[a / x]: B [a / x]}
  \]

\end{frame}

\begin{frame}
  \frametitle{Dependent sum}

  \[
    \inference[$\Sigma$-F]{\Gamma \vdash A \text{ type}\\ \Gamma, x:A \vdash B \text{ type}}{\Gamma \vdash \Sigma x : A. B \text{ type}}
  \]

  \[
    \inference[$\Sigma$-I]{\Gamma \vdash a : A\\ \Gamma\vdash b: B[a / x]}{\Gamma Pair_{[x:A]B} (a,b) : \Sigma x : A. B}
  \]

\end{frame}

\begin{frame}

  \frametitle{Dependent sum cont.}

  \[
    \inference[$\Sigma$-E]{\Gamma, z : \Sigma x: A. B \vdash C \text{ type}\\ \Gamma, x: A, y: B \vdash c : C[Pair_{[x:A]B}(x, y)/z]\\ \Gamma \vdash u : \Sigma x: A. B}
      {R^\Sigma_{[z : \Sigma x: A. B]C}([x : A, y: B]c, u): C[u/z]}
  \]

  \[
    \inference{\Gamma\vdash R^\Sigma_{[z : \Sigma x: A. B]C}([x : A, y: B]c, Pair_{[x:A]B}(a,b)): C[Pair_{[x:A]B}/z]}
      {\Gamma\vdash R^\Sigma_{[z : \Sigma x: A. B]C}([x : A, y: B]c, Pair_{[x:A]B}(a,b)) = c[a/x,b/y] : C [Pair_{[x:A]B}/z}
  \]

\end{frame}

\begin{frame}

  \frametitle{Some definitions}

  Assume $\Gamma \vdash A \text{ type}$, $\Gamma, x : A \vdash B \text{ type}$
  and $\Gamma \vdash u : \Sigma x:A. B$. Then we can define projection
  functions:

  \begin{equation*}
    pu \vcentcolon= R^\Sigma_{[z : \Sigma x: A. B]A}([x : A, y: B]x, u) : A
  \end{equation*}

  \begin{equation*}
    qu \vcentcolon= R^\Sigma_{[z : \Sigma x: A. B]B[pz/x]}([x : A, y: B]y, u) : B[pu]
  \end{equation*}

  If $B$ in $\Pi x: A. B$ and $\Sigma x: A. B$ does not depend on $A$, then we
  simply write:
  \begin{equation*}
    A \to B \vcentcolon= \Pi x: A. B
  \end{equation*}
  \begin{equation*}
    A \times B \vcentcolon= \Sigma x: A. B
  \end{equation*}

  That correspond to ordinary non-dependent functions and products respectively.

\end{frame}

\begin{frame}

  \frametitle{Categories with Families}

  A category with families (CwF) is given by ($\cat{C}$,F) where:

  \begin{itemize}
    \item[$\bullet$] $\cat{C}$ is a category, objects: $\Gamma, \Delta, \dots$ are
      \emph{contexts}, morphisms:  $\sigma, \tau, \dots$ are
      \emph{substitutions} or context morphisms

    \item[$\bullet$] A terminal object $1$ in $\cat{C}$ is the empty context ($\diamond$).

    \item[$\bullet$] $F$ is a functor $F\from \cat{C}^\op \to \fcat{Fam}$; For a context
      $\Gamma$ we write $Ty_F(\Gamma)$ for the indexing set of the family
      $F(\Gamma)$ and its family as $(Ter_F(\Gamma;A) | A \in Ty_F(\Gamma))$;
      $\Gamma \vdash A \text{ type}$ means that $A \in Ty_F(\Gamma)$ and
      $\Gamma \vdash a : A$ means $a \in Ter_F(\Gamma;A)$. For a context
      morphism $\sigma \from \Delta \to \Gamma$ the morphism $F(\sigma)$
      acts on types $\Gamma \vdash A \text{ type}$ as $A\sigma$ and on terms
      $\Gamma \vdash a : A$ as $a\sigma$. The following equations also hold:

    \[
      A1=A \quad (A\sigma)\tau = A(\sigma\tau) \quad a1=a \quad (a\sigma)\tau=a(\sigma\tau)
    \]

  \end{itemize}

\end{frame}

\begin{frame}
  \frametitle{Categories with Families cont.}

  \begin{itemize}
    \item[$\bullet$] There is also the operation of \emph{context extension}:
      if $\vdash \Gamma \text{ context}$ and $\Gamma \vdash A$, there's a
      context $\Gamma.A$, a context morphism $p: \Gamma.A \to \Gamma$, and a
      term $\Gamma.A \vdash q : A p$. This should satisfy the following
      property: for each $\vdash \Delta \text{ context}$ and a context morphism
      $\sigma \from \Delta \to \Gamma$ and $\Delta \vdash a : A \sigma$, there
      is a context morphism $(\sigma, a): \Delta \to \Gamma.A$ such that:
    \[
      p(\sigma, a) = \sigma \quad q(\sigma, a) = a \quad (\sigma, a) \tau =
      (\sigma \tau, a \tau) \quad (p, q) = 1
    \]
    Where in the last equation $1: \Gamma.A \to \Gamma.A$
  \end{itemize}
\end{frame}

\begin{frame}
  \frametitle{Presheaf model}

  Let $\cat{C}$ be a small category. In the following objects of $\cat{C}$ will
  be denoted as $I,J,K$ and morphisms as $f,g,h$. The notation for
  $\ftrcat{\cat{C}^\op}{\Set}$ will be $Psh(\cat{C})$.

  The context $\Gamma$ is then a presheaf on $\cat{C}$, $\Gamma \from
  \cat{C}^\op \to \Set$, that is, we're given a set $\Gamma(I)$ for each
  $I$ in $\cat{C}$, and functions $\Gamma(I) \to \Gamma(J)$, $\rho \mapsto \rho
  f$ (called \emph{restriction} and written as $f$ acting on the right) for each
  $f: J \to I$ such that
  \[
    \rho 1 = \rho \quad \text{and} \quad (\rho f) g = \rho (f g)
  \]
  for $g: K \to J$ and $f: J \to I$. We write $\Gamma_f$ for $\rho \mapsto \rho
  f$.
\end{frame}

\begin{frame}

  \frametitle{Presheaf model cont.}

  The empty context then is a terminal presheaf and a context morphism $\sigma
  \from \Delta \to \Gamma$ is a natural transformation making the diagram
  commute.

  \[
    \xymatrix{
      \Delta(I) \ar[r]^{\sigma_I} \ar[d]^{\Delta_f} & \Gamma(I) \ar[d]^{\Gamma_f} \\
      \Delta(J) \ar[r]^{\sigma_J} & \Gamma(J)
    }
  \]

  If we write $\Gamma_f$ and $\Delta_f$ and $f$ acting on the right, the
  equation becomes:
  \[
    (\sigma \rho) f = \sigma (\rho f)
  \]
  for $\rho$ in $\Delta(I)$.

\end{frame}

\end{document}


% vim mode line:
% vim: ts=2 sts=2 sw=2 expandtab
