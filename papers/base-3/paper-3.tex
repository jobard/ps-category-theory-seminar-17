\def\pathToRoot{../../}\input{\pathToRoot headers/header_lecturenotes}
% \input{\pathToRoot headers/theoremenvts.tex}

\newcommand{\catA}[0]{\cat{A}}
\newcommand{\catB}[0]{\cat{B}}

\newcounter{counter}[section]
\newtheorem{defn}[counter]{Definition}
\newtheorem{exmp}[counter]{Example}
\newtheorem{lemma}[counter]{Lemma}


\title{Natural transformations}
\author{Joachim Bard, Maximilian Wuttke, Nikita Ziuzin}

\begin{document}

\maketitle


\section*{Motivation, definitions and examples}
Morphisms are maps between objects of a category, functors are maps between categories, and finally, natural transformations are maps between functors of the same domain and codomain.

For example, concider a discrete category $\catA$ and an arbitrary category $\catB$.
Now let $\parpair{\catA}{\catB}{F}{G}$ be functors from $\catA$ to $\catB$.
Now because $\catA$ is discrete, we can ignore the morphism-mappings of $F$ and $G$ (since there are only identity morphisms in $\catA$ that are mapped to identity morphisms in $\catB$) and write $F$ and $G$ as families
of objects in $\catB$
$\bigl(F(A)\bigr)_{A\in\catA}$ and $\bigl(G(A)\bigr)_{A\in\catA}$.
When we have a mapping $\alpha$ of $\catB$-morphisms
$\bigl(F(A)\toby{\alpha_A}G(A)\bigr)_{A\in\catA}$
this $\alpha$ could be regarded as a mapping between the functors $F$ and $G$.
Indeed this construction is a special case of the more general definition of natural transformations.

We first define what a natural transformation in general is precisely:
\begin{defn}[Natural transformation]
  \label{def:nat}
  Let $\catA$ and $\catB$ be categories and $\parpair{\catA}{\catB}{F}{G}$ be two functors from $\catA$ to $\catB$.
  Let $\alpha$ be a family of maps
  $\bigl(F(a)\toby{\alpha_A}G(A)\bigr)_{A\in\catA}$.
  Then we define that $\alpha$ is a \demph{natural transformation} from $F$ to $G$ iff
  for every objects and maps $A, A' \in \catA, f \from A \to A'$, the sqare
  \begin{equation}
    \label{eq:nat}
    \begin{array}{c}
      \xymatrix{
        F(A) \ar[r]^{F(f)} \ar[d]_{\alpha_A} & F(A') \ar[d]^{\alpha_{A'}} \\
        G(A) \ar[r]_{G(f)} & G(A')
      }
    \end{array}
  \end{equation}
  commutes.
  The maps $\alpha_A$ are then called the \demph{components} of $\alpha$.
  The commuting square (\ref{eq:nat}) is also called the \demph{naturality axiom} (throught this is not a axiom in logical sense but a defined property for some $\alpha$).
\end{defn}

If $\alpha$ is a natural transformation from $F$ to $G$ we can also write this as a diagram:
\[\xymatrix{
  \catA \rtwocell^F_G{\alpha} & \catB
}\]

\begin{exmp}[Discrete Categories]
  For the first example, the mapping
  $\bigl(F(A)\toby{\alpha_A}G(A)\bigr)_{A\in\catA}$
  the naturality axiom (\ref{eq:nat}) holds, because $\catA$ is discrete, so every morphism $f \from A \to A'$ in $\catA$ is an identy morphism (hence $A=A'$). The diagramm
  \[
    \xymatrix{
      F(A) \ar[r]^{F(I)} \ar[d]_{\alpha_A} & F(A) \ar[d]^{\alpha_{A}} \\
      G(A) \ar[r]_{G(I)} & G(A)
  }\]
  commutes trivially for every identity morphism $I \from A \to A$. Hence $\alpha$ is a natural transformation according to definition~\ref{def:nat}.
\end{exmp}

\begin{exmp}[Determinants and commutative rings]
  Let $n\in\nat$ and $R$ be a commutative ring.
  We have a forgetful functor $U \from \CRing \to \Mon$ by forgetting the addition of the commutative ring.
  Furthermore the set of all $n \times n$ matrices forms a monoid under matrix multiplication.
  If we map $R \in \CRing$ to $R^{n \times n}$ and ring homomorphisms $f \from R \to S$ to
  $\lambda A \in R^{n \times n}. \bigl( f(A_{i,j}) \bigr)_{1 \le i, j \le n}$,
  i.~e. applying $f$ element-wise to the matrix $A$, we get another functor $M_n \from \CRing \to \Mon$.
  We will now see that the \emph{determinant} is a natural transformation between this two functors $M_n$ and $U$.

  First we note that for every ring the determinant of a matrix over this ring is defined canonically.
  Furthermore by definition if $A \in R^{n \times n}$ then $\det A \in R$ for every ring $R$.
  We therefore can establish a family of mappings $\det$:
  $\bigl(M_n(R)\toby{\det_R}U(R)\bigr)_{R\in\CRing}$, because $U(R) = R$.
  Now, to show that $\det$ is indeed a natural transformation, we have to show that the following diagramm commutes for every ring homomorphism $f \from R \to S$:
  \[\xymatrix{
    M_n(R) \ar[r]^{M_n(f)} \ar[d]_{\det_R} & M_n(S) \ar[d]^{\det_S} \\
    U(R) \ar[r]_{U(f)} & U(s)
  }\]
  By unfolding the definitions of $U$ and $M_n$ and applying functional extentionality, this equivalent to
  $f(\det_R(f(A))) = \det_S (\bigl( f(A_{i,j}) \bigr)_{1 \le i, j \le n})$
  for every $A \in R^{n \times n}$.
  This already is a standard result of linear algebra.
  Hence $\det$ is really a natural transformation between the functors $M_n$ and $U$. Or, written as a diagram:
  \[\xymatrix{
    \CRing \rtwocell<5>^{M_n}_{U}{\alpha} & \Mon
  }\]
\end{exmp}


\section{Equivalence of Categories}
\begin{exmp}
  \[
    \begin{tikzcd}
      \catA: & & \catB:\\
      A \arrow[loop left] & & B \arrow[bend left]{d} \arrow[loop left] \\
                          & & B' \arrow[bend left]{u} \arrow[loop left]
    \end{tikzcd}
  \]
\end{exmp}
%todo explain example

\begin{defn}
  \label{def:eqv}
  Tow categories $\catA$ and $\catB$ are \emph{equivalent} if:
  \begin{itemize}
    \item There are functors $F \from \catA \to \catB$ and $G \from \catB \to \catA$ and
    \item there are natural transformations $\eta \from 1_{\catA} \to G \circ F$ and $\varepsilon \from F \circ G \to 1_\catB$.
  \end{itemize}
  The functors $F$ and $G$ are called \emph{equivalences}.
  We write $\catA \eqv \catB$ if $\catA$ and $\catB$ are equivalent.
\end{defn}

\begin{defn}
  \label{def:surj-on-obj}
  A functor $F \from \catA \to \catB$ is \emph{essentially surjective on objects}
  if for every object $B \in \catB$ there is an object $A \in \catA$ such that $F(A) \iso B$.
\end{defn}

\begin{lemma}
  \label{lem:equiv}
  A functor $F \from \catA \to \catB$ is an equivalence if and only if $F$ is full, faithful and surjective on objects.
\end{lemma}

\begin{exmp}
  Let $K$ be a field.
  We define the category $\catA$ such that objects are of the form $K^n$ for $n \in \NN$ and morphisms are linear functions.
  Moreover we denote the category of finite dimensional vector spaces by $\FDVect_K$.
  Objects in this category are finite dimensional vector spaces and morphisms are linear functions.
  Let $F \from \catA \to \FDVect_K$ be the embedding functor from $\catA$ into $\FDVect_K$, i.e. $F$ maps objects and morphisms to themselves.
  Clearly $F$ is full and faithful.
  For surjectivity on objects we fix an arbitrary object $V \in \FDVect_K$.
  Let $n \in \NN$ be the dimension of $V$.
  From linear algebra we know that $V \iso K^n = F(K^n)$.
  Thus $F$ is essentially surjectivity on objects.
  By lemma~\ref{lem:equiv} we obtain that $\catA \eqv \FDVect_K$.
  %In order to show that $\catA \eqv \FDVect_K$ we have to define a functor $F \from \catA \to \FDVect_K$ which is full, faithful and surjective on objects.
\end{exmp}


\section{Horizontal Composition}

\[
\xymatrix@C+.5em{
\cat{A} \rtwocell<4>^F_G{\alpha}   &
\cat{A}' \rtwocell<4>^{F'}_{G'}{\alpha'}   &
\cat{A}''
}
\]
\[
\xymatrix@C+.5em{
  \cat{A} \rtwocell<4>^{F' \of F}_{G' \of G} &\cat{A}'',
}
\]

This horizontal composition is written $\alpha' * \alpha$.
The following diagram commutes because $\alpha'$ is a natural transformation.
\[
  \begin{tikzcd}
    F'(F(A)) \arrow{r}{F'(\alpha_A)} \arrow{d}{\alpha'_{F(A)}} \arrow{dr}{(\alpha' * \alpha)_A} & F'(G(A)) \arrow{d}{\alpha'_{G(A)}} \\
    G'(F(A)) \arrow{r}{G'(\alpha_A)}                                                            & G'(G(A))
  \end{tikzcd}
\]
So we can define $(\alpha' * \alpha)_A$ either as $G'(\alpha_A) \circ \alpha'_{F(A)}$ or as $\alpha'_{G(A)} \circ F'(\alpha_A)$.
Now we want to show that $\alpha' * \alpha$ is again a natural transformation.
Therefore let $f\from A \to A'$ be a morphism in $\catA$.
\[
  \begin{tikzcd}
    F'(F(A)) \arrow{r}{F'(F(f))} \arrow{d}{\alpha'_{F(A)}} & F'(F(A')) \arrow{d}{\alpha'_{F(A')}} \\
    G'(F(A)) \arrow{r}{G'(F(f))} \arrow{d}{G'(\alpha_A)}   & G'(F(A')) \arrow{d}{G'(\alpha_{A'}}  \\
    G'(G(A)) \arrow{r}{G'(G(f))}                           & G'(G(A'))
  \end{tikzcd}
\]

%todo example

We can also combine horizontal and vertical composition.
\[
\xymatrix{
\cat{A}
\ruppertwocell<8>^F{\alpha}
\ar[r]|G
\rlowertwocell<-8>_H{\beta} &
\cat{A}'
\ruppertwocell<8>^{F'}{\hspace{.2em}\alpha'}
\ar[r]|{G'}
\rlowertwocell<-8>_{H'}{\hspace{.2em}\beta'} &
\cat{A}''
}
\]

%todo interchange law

\end{document}

% vim: ts=2 sts=2 sw=2 expandtab
