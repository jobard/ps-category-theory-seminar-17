\def\pathToRoot{../../}\input{\pathToRoot headers/uebungHeader}

\DeclareMathOperator{\A}{\mathscr A}
\DeclareMathOperator{\B}{\mathscr B}
\DeclareMathOperator{\M}{\mathscr M}
\DeclareMathOperator{\Sc}{\mathscr S}

\begin{document}

% Use Basis x or Talk x, where x is the number of the session
\uebunghead{Basis 5}{Adjunctions 2}

\author{Mostafa Abouhamra}

\begin{hint}
Read Chapters 2.2 and 2.3.
\end{hint}

\section{Units and Counits}


\begin{exercise}
\begin{enumerate}
\item Show that for any adjunction, the right adjoint is full and faithful if and only if the counit is an isomorphism.
\item An adjunction satisfying the equivalent conditions of part (a) is called a reflection. Can you give an example of a reflection?
\end{enumerate}
\end{exercise}

\begin{exercise}
Given a functor $F: \A \rightarrow \B$ and a category $\Sc$, there is a functor $F^*: [\B,\Sc] \rightarrow [\A, \Sc]$ defined on objects $Y \in [\B,\Sc]$ by $F^*(Y) = Y \of F$ and on maps $\alpha$ by $F^*(\alpha) = \alpha F$. Show that any adjunction
\begin{tikzcd}[ampersand replacement=\&, every label/.append style = {font = \footnotesize}]
        \A \arrow[r, shift left,"F"]
        \& \B \arrow[l, shift left, "G"]
\end{tikzcd}
with  $F \dashv G$ and category $\Sc$ give rise to an adjunction
\begin{tikzcd}[ampersand replacement=\&, every label/.append style = {font = \footnotesize}]
        [\A,\Sc] \arrow[r, shift left,"G^*"]
        \&  \arrow[l, shift left, "F^*"]  [\B,\Sc]
\end{tikzcd}
with $G^* \dashv F^*$.
\end{exercise}

\section{Adjunctions by Initial Objects}

\begin{exercise}
Given two functors $F: \A \rightarrow \B$ and $G: \B \rightarrow \A$ and a natural transformation $\eta: 1_{\A} \rightarrow GF$ such that $\eta_A: A \rightarrow GF(A)$ is initial in ($A \Rightarrow G$) for every $A \in \A$, prove there is an adjunction
\begin{tikzcd}[ampersand replacement=\&, every label/.append style = {font = \footnotesize}]
        \A \arrow[r, shift left,"F"]
        \& \B \arrow[l, shift left, "G"]
\end{tikzcd}
with $F \dashv G$.
\end{exercise}

\begin{exercise}
Give an example of an adjunction and formalize it in 3 different ways:
\begin{enumerate}
\item Define the adjunction according to definition 2.2.1.
\item Give the unit and counit. Why do they satisfy the triangle identities?
\item Define the adjunction via the initial object.
\end{enumerate}
\end{exercise}


\begin{exercise}
What can be said about adjunctions between groups (regarded as one-object categories)?
\end{exercise}

\textbf{Fact}: Let $G: \B \rightarrow \A$ be a functor. Then G has a left adjoint if and only if for each $A \in \A$, the category ($A \Rightarrow G$) has an initial object.\\

\begin{exercise}
State and prove the dual of the above fact.
\end{exercise}

\begin{exercise}
  Recall Definition 1.3.15 of the script, which states that an equivalence of categories $\A$ and $\B$ consists of two functors $F$ and $G$ and two natural isomorphisms $\eta$ and $\epsilon$ which relate their compositions to the respective identity functors.
\begin{enumerate}
\item Give an example of an adjunction that is not an equivalence.
\item Let ($F, G, \eta, \epsilon$) be an equivalence of categories. Prove that F is left adjoint to G.
\end{enumerate}
\end{exercise}

\end{document}
