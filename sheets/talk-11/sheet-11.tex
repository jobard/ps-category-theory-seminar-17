\def\pathToRoot{../../}\input{\pathToRoot headers/uebungHeader}

\begin{document}

% Use Basis x or Talk x, where x is the number of the session
\uebunghead{Talk 11}{Dependent Types}
\author{Nikita Ziuzin}

\begin{hint}
  Read Martin Hofmann's Syntax and Semantics of Dependent Types and section $1$
  in Simon Huber's A Model of Type Theory in Cubical Sets.
\end{hint}

\begin{exercise}
  Read Hofmann's section on type for natural numbers (section 2.1.3).

  Familiarize yourself with identity types (ibid., section 2.1.5)

  By analogy to the type of natural numbers define the rules for a list type
  former which to any type $A$ associates a type $List(A)$ consisting of finite
  sequences of elements of $A$.

  \emph{Hint}: think of lists as inductively
  generated from the empty list by successive additions of elements of $A$
  ("cons").

  Define a length function of type $List(A) \to \mathbb{N}$ and
  define a type $Vec_A(n)$ of lists of length $n$ for each $n : \mathbb{N}$
  using lists, the identity type and the $\Sigma$-type.
\end{exercise}

\begin{exercise}
  After reading the material for the previous exercise, read a section on
  universes in Hofmann's notes (2.1.6).

  Give the rules for universe $U$ containing a code for $\hat{0}$ for the empty
  type and a code $\hat{1}$ for the unit type $1$. Show that in a type theory
  which supports natural numbers, this universe, and the empty type itself the
  following type in the empty context is inhabited
  \[
    \diamond \vdash Id_\mathbb{N}(0, Suc(0)) \to 0 \text{ type}
  \]
  corresponding to Peano's fourth axiom $1 \neq 0$.

  \emph{Hint}: define using $R^\mathbb{N}$ a function $f \from \mathbb{N} \to
  U$ such that $\diamond \vdash f 0 = \hat{1}: U$ and $\diamond \vdash
  f(Suc(0)) = \hat{0} : U$
\end{exercise}

\begin{exercise}
  Recall the notation for the category of elements of a presheaf $\Gamma$:
  $\int_\cat{C}\Gamma$ Check that substitution of $A\from
  (\int_\cat{C}\Gamma)^\op \to \Set$ with $\sigma \from \Delta \to \Gamma$
  corresponds to precomposing with $\int_\cat{C} \sigma : \int_\cat{C} \Delta
  \to \int_\cat{C} \Gamma$ induced by $\sigma$.

  Then verify that this construction imposes a functor $\int_\cat{C} \from
  Psh(\cat{C}) \to \CAT$.
\end{exercise}
\begin{answer}
\end{answer}

\begin{exercise}
  Define dependent sum using presheaf model notation

  \emph{Hint}: Read section $1.2$ in Huber's thesis to recall presheaf model
  construction. Try to come up with your definition before checking the answer
  given in the section.
\end{exercise}
\begin{answer}
  Section $1.2.2$.
\end{answer}

\end{document}

%%% Local Variables:
%%% mode: latex
%%% TeX-master: t
%%% End:
