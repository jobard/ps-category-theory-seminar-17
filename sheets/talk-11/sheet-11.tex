\def\pathToRoot{../../}\input{\pathToRoot headers/uebungHeader}

\begin{document}

% Use Basis x or Talk x, where x is the number of the session
\uebunghead{Talk 11}{Dependent Types}
\author{Nikita Ziuzin}

\begin{hint}
  Read sections $2.1.3, 2.1.5, 2.1.6$ in Martin Hofmann's \href{http://www.irif.fr/~mellies/mpri/mpri-ens/articles/hofmann-syntax-and-semantics-of-dependent-types.pdf}{Syntax and Semantics of Dependent Types} and section $1$ in Simon Huber's \href{http://www.cse.chalmers.se/~simonhu/misc/lic.pdf}{A Model of Type Theory in Cubical Sets}.
\end{hint}

\begin{exercise}
  Define the sets $\mathit{Ty}_F(\Gamma)$ and $\mathit{Ter}_F(\Gamma, A)$ for the presheaf model.
\end{exercise}
\begin{answer}
  $\mathit{Ty}_F(\Gamma)$ is the first projection in $F(\Gamma)$, that is a set with some
  objects (possibly empty).
  \[
    \mathit{Ty}_F(\Gamma) := \{A_\Gamma, B_\Gamma, \dots\}
  \]
  $\mathit{Ter}_F(\Gamma, A)$ is then an element of the second projection of $F(\Gamma)$,
  that is a member of the family of sets indexed over $\mathit{Ty}_F(\Gamma)$, which has
  A as its index and also may be empty.
  \[
    \mathit{Ter}_F(\Gamma, A) := \{a_A, b_A, \dots\}_A
  \]
\end{answer}

\begin{exercise}
  Let $(\cat C, \mathcal F)$ be a category with families.
  Check that the following square is a pullback in $\cat{C}$:
  \[
    \begin{tikzcd}
      \Delta.A\sigma \arrow{r}{(\sigma \circ p_\Delta, q)} \arrow{d}{p_\Delta} & \Gamma.A \arrow{d}{p_\Gamma}\\
      \Delta \arrow{r}{\sigma} & \Gamma
    \end{tikzcd}
  \]
  where $\Delta.A \vdash q : A p_\Delta$
\end{exercise}

\begin{exercise}
  Recall the notation $\int_\cat{C}\Gamma$ for the category of elements of a presheaf $\Gamma$ (section $1.2$ p.$16$ in Huber's thesis).
  Check that substitution in $A\from (\int_\cat{C}\Gamma)^\op \to \Set$ with $\sigma \from \Delta \to \Gamma$ corresponds to precomposing with $\int_\cat{C} \sigma : \int_\cat{C} \Delta \to \int_\cat{C} \Gamma$ induced by $\sigma$.

  Then verify that this construction imposes a functor $\int_\cat{C} \from \mathit{Psh}(\cat{C}) \to \CAT$.
\end{exercise}
\begin{answer}
\end{answer}

\begin{exercise}
  From the slides we know that if $\Gamma \vdash A, B ~\text{type}$, i.e. $B$ does not depend on $A$,
  then $\Pi x: A. B$ corresponds to the ordinary non-dependent function type:
  \begin{equation*}
    A \to B \vcentcolon= \Pi x: A. B
  \end{equation*}

  Give a term of type $A \to A$, and the elements in $\mathit{Ty}(\Gamma)$ and $\mathit{Ter}(\Gamma, A \to A)$ corresponding to the type and the term.

  \emph{Hint}: Read section $1.2.1$ in Huber's thesis for the general case.

\end{exercise}

\begin{exercise}
  Suppose we have two types $\Gamma \vdash A,B \text{ type}$ each not depending on the other.
  Then we can define a product type in the presheaf model in the following way:
  \[
    (A \times B)\rho := \{(a,b) | a \in A\rho \land b \in B\rho\}
  \]

  Now suppose that we have dependent types $\Gamma \vdash C \text{ type}$ and
  $\Gamma, C \vdash D \text{ type}$.

  Define the product type that involves a
  dependent type (i.e. the dependent sum type) in the presheaf model.

  \[
    (\Sigma AB)\rho := ???
  \]

  \emph{Hint}: The definition is \emph{almost} identical to the one given
  above. But what do you need to change? To recall the presheaf model, read
  section $1.2$ in Huber's thesis.
\end{exercise}
\begin{answer}
  Section $1.2.2$.
\end{answer}

\begin{exercise}
  Read Hofmann's section on the type of natural numbers (section 2.1.3) and familiarize yourself with identity types (ibid., section 2.1.5)

  By analogy to the type of natural numbers define the rules for a list type
  former which to any type $A$ associates a type $\mathit{List}(A)$ consisting of finite
  sequences of elements of $A$.

  \emph{Hint}: think of lists as inductively
  generated from the empty list by successive additions of elements of $A$
  ("cons").

  Define a length function of type $\mathit{List}(A) \to \mathbb{N}$ and
  define a type $\mathit{Vec}_A(n)$ of lists of length $n$ for each $n : \mathbb{N}$
  using lists, the identity type and the $\Sigma$-type.
\end{exercise}

\begin{exercise}
  Read the section on universes in Hofmann's notes (2.1.6).

  Give the rules for universe $U$ containing a code for $\hat{0}$ for the empty
  type and a code $\hat{1}$ for the unit type $1$. Show that in a type theory
  which supports natural numbers, this universe, and the empty type itself the
  following type in the empty context is inhabited
  \[
    \diamond \vdash Id_\mathbb{N}(0, Suc(0)) \to 0 \text{ type}
  \]
  corresponding to Peano's fourth axiom $1 \neq 0$.

  \emph{Hint}: define using $R^\mathbb{N}$ a function $f \from \mathbb{N} \to
  U$ such that $\diamond \vdash f 0 = \hat{1}: U$ and $\diamond \vdash
  f(Suc(0)) = \hat{0} : U$
\end{exercise}

\end{document}

%%% Local Variables:
%%% mode: latex
%%% TeX-master: t
%%% End:
