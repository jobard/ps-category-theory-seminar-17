\def\pathToRoot{../../}\input{\pathToRoot headers/uebungHeader}

\usepackage{semantic}
\usepackage{todonotes}


\begin{document}

% Use Basis x or Talk x, where x is the number of the session
\uebunghead{Talk 9}{Cartesian Closed Categories as a model for the STLC}
\author{Joachim Bard}

\begin{hint}
    Read Andrew Pitt's lecture notes on categorical semantics for the STLC.
\end{hint}

\begin{exercise}
    Let $\cat{C}$ be a cartesian closed category and let $M$ map ground types and constants into $\cat{C}$ as defined in chapter 4 of Andrew Pitt's lecture notes.
    What morphism does $M |[x: A, y: B \vdash \lambda f: A \to B.\,\mathsf{fst}(f\,x, y) : (A \to B) \to B|]$ correspond to?
    Simplify your answer as much as possible.
\end{exercise}
\begin{answer}
    Define $\Gamma := x: A, y: B, f: A \to B$.
    \begin{align*}
          & M |[x: A, y: B \vdash \lambda f: A \to B.\,\mathsf{fst}(f\,x, y) : (A \to B) \to B|]\\
        = &\, \mathsf{cur}(M |[x: A, y: B,f: A \to B \vdash \mathsf{fst}(f\,x, y) : B|])\\
        = &\,\mathsf{cur}(\pi_1 \circ M|[x: A, y: B, f: A \to B \vdash (f\,x, y) : B \times B|])\\
        = &\, \mathsf{cur}(\pi_1 \circ (M|[x: A, y: B,f: A \to B \vdash f\,x : B|], M|[x: A, y: B,f: A \to B \vdash y: B|]))\\
        = &\, \mathsf{cur}(\pi_1 \circ (\mathsf{app} \circ (M|[x: A, y: B,f: A \to B \vdash f : A \to B|], M|[\Gamma \vdash x: A|]), M|[x: A, y: B \vdash y: B|] \circ \pi_1))\\
        = &\, \mathsf{cur}(\pi_1 \circ (\mathsf{app} \circ (\pi_2, M|[x:A, y:B \vdash x:A|] \circ \pi_1), \pi_2 \circ \pi_1))\\
        = &\, \mathsf{cur}(\pi_1 \circ (\mathsf{app} \circ (\pi_2, M|[x:A \vdash x:A|] \circ \pi_1 \circ \pi_1), \pi_2 \circ \pi_1))\\
        = &\, \mathsf{cur}(\pi_1 \circ (\mathsf{app} \circ (\pi_2, \pi_2 \circ \pi_1 \circ \pi_1), \pi_2 \circ \pi_1))\\
        = &\, \mathsf{cur}(\mathsf{app} \circ (\pi_2, \pi_2 \circ \pi_1 \circ \pi_1))
    \end{align*}
\end{answer}

\begin{exercise}
    \label{ex:exp}
    \begin{itemize}
        \item[a)] Let $A, B, C$ be simple types.
            Define lamda terms $cas$ and $car$ with the following types and $\beta\eta$-equivalences:
            \begin{align*}
                & \vdash cas: (A \times B \to C) \to A \to B \to C \\
                & \vdash car: (A \to B \to C) \to A \times B \to C\\
                & f: A \times B \to C \vdash car\, (cas\, f) =_{\beta\eta} f : A \times B \to C \\
                & g: A \to B \to C \vdash cas\, (car\, g) =_{\beta\eta} g : A \to B \to C
            \end{align*}
        \item[b)] Let $\cat{C}$ be an arbitrary cartesian closed category and $X, Y, Z \in \cat{C}$.
            Use part a) to show that $Z^{X \times Y} \iso (Z^Y)^X$.
    \end{itemize}
\end{exercise}
\begin{answer}
    \begin{itemize}
        \item[a)] We define $cas$ and $car$ as follows:
            \begin{align*}
                & cas := \lambda f: A \times B \to C.\,\lambda x: A.\,\lambda y: B.\,f(x, y) \\
                & car := \lambda g: A \to B \to C.\,\lambda p: A \times B.\,g\,(\mathsf{fst}\,p)\,(\mathsf{snd}\,p)
            \end{align*}
            It is clear that $cas$ and $car$ have the demanded types.
        \item[b)] We consider the internal language of $\cat{C}$.
            \todo{todo}
    \end{itemize}
\end{answer}

\begin{exercise}
    Let $X$ be an object in a cartesian closed category and let $1$ be the terminal object.
    Use a similar construction to exercise \ref{ex:exp} to show $X^1 \iso X$.
\end{exercise}
\begin{answer}
    %todo
\end{answer}

\begin{exercise}
    Recall the soundness theorem:\\
    If $\Gamma \vdash s =_{\beta\eta} t : A$, then $M |[ \Gamma \vdash s: A |] = M |[ \Gamma \vdash t: A |]$.\\
    It is proven by induction on $\beta\eta$-equivalence.
    Show the case for $\inference{\Gamma \vdash s: A \times B}{\Gamma \vdash s =_{\beta\eta} (\mathsf{fst}\,s, \mathsf{snd}\,s): A \times B}$.
\end{exercise}
\begin{answer}
    We have to show $M |[ \Gamma \vdash s: A \times B |] = M |[ \Gamma \vdash (\mathsf{fst}\,s, \mathsf{snd}\,s): A \times B |]$.
    We start with the right side and simplify:
    \begin{align*}
          & M|[\Gamma \vdash (\mathsf{fst}\,s, \mathsf{snd}\,s): A \times B |]\\
        = \,& (M|[ \Gamma \vdash \mathsf{fst}\,s: A|], M|[\Gamma \vdash \mathsf{snd}\,s: B|])\\
        = \,& (\pi_1 \circ M|[ \Gamma \vdash s: A \times B |], \pi_2 \circ M|[\Gamma \vdash s: A \times B|])\\
        = \,& M|[ \Gamma \vdash s: A \times B |]
    \end{align*}
    \todo{show diagram}
\end{answer}

\begin{exercise}
    \label{ex:pure-peirce}
    A term in the lambda calculus is \textit{pure} if it contains no constants.
    Show that there is no \textit{pure} term $s$ such that $ \vdash s: ((G \to H) \to G) \to G$ for two different ground types $G$ and $H$.\\
    Hint: $([0, 1], \le)$ regarded as a category is cartesian closed.
    What is the exponential $Y^X$ of $X$ and $Y$ where $X, Y \in [0, 1]$?
\end{exercise}
\begin{answer}
    The exponential $Y^X$ is defined as:
    \begin{align*}
        Y^X := \begin{cases}
            1, \text{ if } X \le Y\\
            Y, \text{ if } Y < X
               \end{cases}
    \end{align*}
    We write $X -> Y$ for $Y^X$.
    Let us assume a pure term $s$ such that $ \vdash s: ((G \to H) \to G) \to G$ for two different ground types $G$ and $H$.
    We define $M$ as:
    \begin{align*}
        M(G') := \begin{cases}
                \frac{1}{2}, \text{ if } G' = G\\
                0, \text{ else}
            \end{cases}
    \end{align*}
    We obtain for $M|[\ \vdash s: ((G \to H) \to G) \to G|]$:
    \begin{align*}
            & 1 \le M|[((G \to H) \to G) \to G|] \\
        <=>\, & 1 \le M|[(G \to H) \to G|] -> M|[G|] \\
        <=>\, & 1 \le (M|[G \to H|] -> M|[G|]) -> M|[G|] \\
        <=>\, & 1 \le ((M|[G|] -> M|[H|]) -> M|[G|]) -> M|[G|] \\
        <=>\, & 1 \le ((\frac{1}{2} -> 0) -> \frac{1}{2}) -> \frac{1}{2} \\
        <=>\, & 1 \le (0 -> \frac{1}{2}) -> \frac{1}{2} \\
        <=>\, & 1 \le 1 -> \frac{1}{2} \\
        <=>\, & 1 \le \frac{1}{2}
    \end{align*}
    But $1 \nleq \frac{1}{2}$ and therefore the assumption of the existence of such a pure term $s$ is false.
\end{answer}

\begin{definition}{Cartesian Closed Pre-order}
    Let $(X, \le)$ be a pre-ordered set.
    The corresponding category is cartesian closed iff it has:
    \begin{itemize}
        \item a greatest element $\top$ such that $\forall x \in X: x \le \top$,
        \item binary meets $x \land y$ such that $\forall z \in X: z \le x \land y <=> z \le x \land z \le y$ and
        \item Heyting implications $x -> y$ such that $\forall z \in X: z \le x -> y <=> z \land x \le y$.
    \end{itemize}
\end{definition}

\begin{definition}{Intuitionistic Propositional Logic (IPL)}
    Formulas in IPL are defined on the following grammar:
    \[\varphi, \psi ::= p \mid \top \mid \varphi \land \psi \mid \varphi => \psi\]
    where $p$ corresponds to propositional variables, $\top$ to the proposition which is always true,
    $\varphi \land \psi$ to conjunction and $\varphi => \psi$ to implication.
    We define an entailment relation $\Phi \vdash \varphi$ where $\varphi$ is a formula and $\Phi$ is a list of formulas:
    \begin{align*}
        & \qquad\quad \inference{}{\Phi, \varphi \vdash \varphi}\qquad\qquad
        \inference{\Phi \vdash \varphi}{\Phi, \psi \vdash \varphi}\qquad\qquad
        \inference{}{\Phi \vdash \top}\\[1em]
        & \inference{\Phi \vdash \varphi \quad \Phi \vdash \psi}{\Phi \vdash \varphi \land \psi}\qquad\qquad
        \inference{\Phi \vdash \varphi \land \psi}{\Phi \vdash \varphi}\qquad\qquad
        \inference{\Phi \vdash \varphi \land \psi}{\Phi \vdash \psi}\\[1em]
        & \qquad\qquad \inference{\Phi, \varphi \vdash \psi}{\Phi \vdash \varphi => \psi}\qquad\qquad
        \inference{\Phi \vdash \varphi => \psi \quad \Phi \vdash \varphi}{\Phi \vdash \psi}
    \end{align*}
\end{definition}

\begin{exercise}
    Let $(X, \le)$ be a cartesian closed pre-order and assume a function $M$ which assigns to each propositional variable $p$ an element of $X$, i.e. $M(p) \in X$.
    \begin{itemize}
        \item[a)] Lift $M$ to arbitrary formulas $\varphi$ by defining $M|[\varphi|] \in X$ and to lists of variables $\Phi$ such that:
            \begin{itemize}
                \item If $\Phi \vdash \varphi$, then $M|[\Phi|] \le M|[\varphi|]$ (Soundness).
                \item If $M|[\Phi|] \le M|[\varphi|]$ holds for every cartesian closed pre-order $\cat{C}$ and every choice of $M$, then $\Phi \vdash \varphi$ (Completeness).
            \end{itemize}
            You don't have to prove the above statements.
        \item[b)] Show that Peirce's law $\emptyset \vdash ((\varphi => \psi) => \varphi) => \varphi$ is not provable in IPL.
            How is this related to exercise \ref{ex:pure-peirce}? Why?
    \end{itemize}
\end{exercise}
\begin{answer}
    \begin{itemize}
        \item[a)]
            \begin{align*}
                M|[p|] & := M(p)\\
                M|[\top|] & := \top && (\text{greatest element})\\
                M|[\varphi \land \psi|] & := M|[\varphi|] \land M|[\psi|] && (\text{binary meet})\\
                M|[\varphi => \psi|] & := M|[\varphi|] -> M|[\psi|] && (\text{Heyting implication})\\
                \\
                M|[\emptyset|] & := \top && (\text{greatest element})\\
                M|[\Phi, \varphi|] & := M|[\Phi|] \land M|[\varphi|] && (\text{binary meet})
            \end{align*}
        \item[b)] Let us consider the cartesian closed pre-order $([0,1], \le)$.
            The Heyting implication $x -> y$ is defined as:
            \begin{align*}
                x -> y := \begin{cases}
                    1, \text{ if } x \le y\\
                    y, \text{ if } y < x
                       \end{cases}
            \end{align*}
            We assume Peirce's law $\emptyset \vdash ((\varphi => \psi) => \varphi) => \varphi$ for two formulas $\varphi$ and $\psi$.
            We define $M$ as:
            \begin{align*}
                M(\varphi') := \begin{cases}
                        \frac{1}{2}, \text{ if } \varphi' = \varphi\\
                        0, \text{ else}
                    \end{cases}
            \end{align*}
            By soundness we obtain:
            \begin{align*}
                      & M|[\emptyset|] \le M|[((\varphi \to \psi) \to \varphi) \to \varphi|] \\
                <=>\, & 1 \le M|[((\varphi \to \psi) \to \varphi) \to \varphi|] \\
                <=>\, & 1 \le M|[(\varphi \to \psi) \to \varphi|] -> M|[\varphi|] \\
                <=>\, & 1 \le (M|[\varphi \to \psi|] -> M|[\varphi|]) -> M|[\varphi|] \\
                <=>\, & 1 \le ((M|[\varphi|] -> M|[\psi|]) -> M|[\varphi|]) -> M|[\varphi|] \\
                <=>\, & 1 \le ((\frac{1}{2} -> 0) -> \frac{1}{2}) -> \frac{1}{2} \\
                <=>\, & 1 \le (0 -> \frac{1}{2}) -> \frac{1}{2} \\
                <=>\, & 1 \le 1 -> \frac{1}{2} \\
                <=>\, & 1 \le \frac{1}{2}
            \end{align*}
            Peirce's law $\emptyset \vdash ((\varphi => \psi) => \varphi) => \varphi$ is not provable in IPL since $1 \nleq \frac{1}{2}$.
            \todo{connection to previous exercise}
    \end{itemize}
\end{answer}

\end{document}

%%% Local Variables:
%%% mode: latex
%%% TeX-master: t
%%% End:
